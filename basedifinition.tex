\section{Понятие идеала в кольце многочленов}
\label{cha:basedifinition}

В первой части научно-исследовательской работы даются базовые определения и понятия, которые будут использоваться на протяжении всей расчетно-пояснительной записки.
\\Пусть $R$ - коммутативное кольцо с единицей $1$.


\subsection{Понятие идеала}

Непустое подмножество $I$ кольца $R$ называется идеалом в $R$ (записывается $I \triangleleft R$), если:

1) для любых элементов $a,b \in I$ элемент $a + b \in I$;

2)для любых $a \in I, c \in R$ элемент $ac \in I$.

Идеал $I$ кольца $R$ называется главным, если существует такой элемент $a \in I$, что $ I = (a)$. Элемент $a$ называется порождающим( или образующим)  для идеала $I$.

Кольцо  $R$ называется кольцом главных идеалов, если каждый идеал кольца $R$ является главным.

\subsection{Базис идеала}

Обобщим по понятие главного идеала $I$ кольца $R$.  Пусть 
$a_{1}, ... , a_{k}$ - произвольные элементы кольца $R$.

Если множество 
$$ (a_{1}, ... , a_{k}) =\left\{ a_{1}r_{1} + ... + a_{k}r_{k};  r_{1}, ... , r_{k} \in R \right\} $$ есть идеал $I$ кольца $R$, тогда говорят, что элементы $a_{1}, ... , a_{k}$ составляют базис идеала $I$. Обозначается  $I = (a_{1}, ... , a_{k})$.

Можно заметить, что при определении базиса идеала нет требования на минимальное количество базисных элементов. Это связано с тем, что при добавлении к базису произвольного элемента идеала, мы получаем тот же самый базис.

Для любого элемента $a \in (a_{1}, ..., a_{k})$:
$$ (a_{1}, .. , a_{k}, a) = (a_{1}, ... ,a_{k}) $$ 

Легко убедиться в достоверности этого утверждения. По определению $ (a_{1}, ... , a_{k}, a) = \left\{ a_{1}r_{1} + ... + a_{k}r_{k} + ar;   r_{1}, ... , r_{k}, r \in R \right\} $. По определению идеала $I$ кольца $R$, если $a \in I, r \in R$, то элемент $ar \in I$. Тогда 
$ar = a_{1}r_{1}' + ... + a_{k}r_{k}';  r_{1}', ... , r_{k}', \in R $. Подставим это выражение в определение базиса $ (a_{1}, ... , a_{k}, a) = \left\{ a_{1}r_{1} + ... + a_{k}r_{k} + a_{1}r_{1}' + ... + a_{k}r_{k}';  r_{1}, r_{1}', ... , r_{k},  r_{k}' \in R \right\} = 
 \left\{ a_{1}(r_{1} + r_{1}') + ... + a_{k}(r_{k} + r_{k}');  r_{1}, r_{1}', ... , r_{k}, r_{k}' \in R \right\} $. Таким образом, при добавлении нового базисного элемента, который принадлежит идеалу, базис не изменится.
 
 \subsection{Идеал в кольцах многочленов}
 
 Пусть $K \left[ x_{1}, ... , x_{n} \right]$ - кольцо многочленов от переменных $x_{1}, ... ,x_{n}$ над полем $\mathcal{K}$.
 Рассмотрим идеалы в кольцах многочленов.
 
 Легко убедиться в том, что кольцо многочленов от нескольких переменных не является кольцом главных идеалов.
В кольце многочленов $K \left[ x_{1}, ... , x_{n} \right]$ выделим множество многочленов, у которых свободных член равен $0$. Данное множество образует идеал в этом кольце, обозначим его $I_{0}$. Пусть $ I_{0}  = (f), f \in K \left[ x_{1}, ... , x_{n} \right] $. Поскольку $x_{1} \in I_{0}$, значит $f$ - либо ненулевая константа (тогда $I_{0} =K \left[ x_{1}, ... , x_{n} \right] $, что является противоречием условия), либо $f = ax, a \in \mathcal{K}, a \neq 0$. Но $x_{2} \in I_{0}$, значит $f$ делит 
 $x_{2}$. Возникает противоречие.
 
Раз кольцо многочленов от нескольких переменных не является кольцом главных идеалов, значит имеет место понятие базиса идеала. В следующем параграфе будут подробно рассмотрены базисы идеала в кольце многочленов от нескольких переменных и выведены некоторые утверждения.







 
 
 


%%% Local Variables:
%%% mode: latex
%%% TeX-master: "rpz"
%%% End:
