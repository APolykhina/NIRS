\section{Мономиальные базисы идеалов в кольцах многочленов} 

В данном разделе будут подробно рассмотрены мономиальные базисы идеалов в кольцах многочленов от нескольких переменных и введено понятие базиса Гребнера.

\subsection{Мономиальный порядок и его свойства}
Пусть $ M_{n} $ - множество мономов. 
Мономиальным упорядочением $ \prec $ на $ M_{n} $ называется линейный порядок, удовлетворяющий свойствам:

1)  $M \prec N \Rightarrow MP \prec NP,  \forall M, N, P \in M_{n}$ 

2)$ 1 \preceq M,  \forall M \in M_{n} $

Пусть $ M = x_{1}^{a_{1}}...x_{n}^{a_{n}} $, $ N = x_{1}^{b_{1}}...x_{n}^{b_{n}} $ - произвольные мономы из $ M_{n} $.
Приведем несколько примеров бинарных отношений на $M_{n}$, которые являются мономиальными упорядочениями:

1.  Лексикографическое упорядочение (lex):
$$ M \prec_{lex} N \Leftrightarrow (a_{1}, ... , a_{n}) \prec_{lex} (b_{1}, ..., b_{n}) $$

2. Сначала по степени, затем лексикографическое упорядочение (deglex):
$$ M \prec_{deglex} N \Leftrightarrow (deg(M), a_{1}, ... , a_{n}) \prec_{lex} (deg(N), b_{1}, ..., b_{n}) $$

3. Сначала по степени, затем обратное лексикографическое упорядочение (degrevlex):
$$ M \prec_{degrevlex} N \Leftrightarrow (deg(M), b_{n}, ... , b_{1}) \prec_{lex} (deg(N), a_{n}, ..., a_{1}) $$

Далее введем несколько обозначений, которые понадобятся нам в дальнейшем.

Пусть $ f= \sum_{a}p_{a}x^{a}$ - ненулевой полином в $K \left[ x_{1}, ... , x_{n} \right]$, и пусть $ \prec $ - мономиальное упорядочение.

1. Мультистепень полинома $f$ определяется так:
$$ multideg(f) = max(a \in Z_{\geq0}^{n}: p_{a} \neq 0) $$

2. Старший коэффициент полинома $f$:
$$ LC(f) = p_{multideg}(f) $$

3. Старший моном полинома $f$:
$$ LM(f) = x^{multideg}(f) $$ ( с коэффициентом 1)

4. Старший член полинома $f$:
$$ LT(f) = LC(f)LM(f) $$ 

\subsection{Мономиальные идеалы}

Введем понятие мономиального идеала в кольцое многочленов от нескольких переменных.

Идеал $I \triangleleft K \left[ x_{1}, ... , x_{n} \right]$ называется мономиальным, если существует подмножество $A \in Z_{\geq0}^{n}$ (которое может быть бесконечым), такое, что $I$ состоит из всех конечных сумм вида $\sum_{a \in A}h_{a}x^{a}$, где $h_{a} \in  K \left[ x_{1}, ... , x_{n} \right]$. Такой идеал будем обозначать $(x^{a}: a \in A)$.

\textbf{Лемма Диксона.} {
\it 
Любой мономиальный идеал $ I = ( x^{a} : a \in A)$ может быть представлен в виде $ I = (x^{a(1)}, ... , x^{a(s)}$, где $a(1), ..., a(s) \in A$. В частности, $I$ имеет конечный базис.
}

\subsection{Теорема Гильберта о базисе}
В предыдущем параграфе было  показано, что любой мономиальный идеал кольца многочленов от нескольких переменных иметь конечный базис. В этом разделе будет доказано, что любой идеал в кольце многочленов(не только мономиальный) имеет конечный базис.

\textbf{Теорема Гильберта о базисе.}{
\it
Каждый идеал $I \triangleleft K \left[ x_{1}, ... , x_{n} \right]$ является конечно порожденным, то есть $I = (g_{1}, ..., g_{s})$, где $g_{1}, ..., g_{s} \in I$.
}

\textbf{Доказательство.} 
Пусть $I \triangleleft K \left[ x_{1}, ... , x_{n} \right]$. Обозначим через $a_{t}$ множество элементов $a \in \mathcal{K}$, которое содержит в себе $LC(f)$, где $f \in I$ и $ multideg(f) = t $.
Легко видно, что $a_{i}$ является идеалом в $ \mathcal{K} $. (Если $a,b \in a_{i}$, то $ a \pm b \in a_{i}$, это легко увидеть, достаточно взять сумму или разность соответсвующих многочленов. Если $ c \in \mathcal{K}$, то $ca \in a_{i}$, это можно увидеть при умножении соответсвующего многочлена на $c$.)
Кроме того, имеем $$ a_{0} \subset a_{1} \subset a_{2} ... $$
другими словами последовательность идеалов $\left[ a_{i} \right]$ возрастающая. 
\\Пусть последовательность идеалов стабилизируется на $a_{r}$:
$$ a_{0} \subset a_{1} \subset a_{2} ... \subset a_{r} = a_{r+1} = ...$$
Пусть 
$$  a_{0}  = (a_{01}, a_{02}, ..., a_{0n_{0}})$$
$$  a_{r}  = (a_{r1}, a_{r2}, ..., a_{rn_{0}})$$
Для каждого $ i = 0, ...., r$ и $j = 0, ..., n_{i}$ пусть $f_{ij}$ - многочлен из $I$ степени $i$ со старшим коэффициентом $a_{ij}$. Покажем, что многочлены $f_{ij}$ образуют базис для $I$.
\\ Пусть $f$ - многочлен степени $d$ из $I$. Индукцией по $d$ докажем, что $f$ лежит в идеале, порожденном $f_{ij}$.
\\ Если $d > r$, то заметим, что старшие коэффициенты многочленов 
$$ X^{d-r}f_{r1}, ... , X^{d-r}f_{rn_{r}} $$
порождают $a_{d}$. Значит, существуют $ c_{1}, ... , c_{n_{r}} \in  \mathcal{K}$, такие, что многочлен
$$ f - c_{1}X^{d-r}f_{r1} - ... - c_{n_{r}}X^{d-r}f_{rn_{r}} $$
имеет степень $ < d$. Этот многочлен лежит в $I$.
Если $ d \leq r$, мы можем получить многочлен степени $ < d$, лежащий в $I$, вычитая некоторую линейную комбинацию
$$ f - c_{1}f_{d1} - ... - c_{n_{r}}f_{dn_{r}} $$
По индукции мы можем найти такой многочлен $g \in I$, который порожден $f_{ij}$ и $f - g = 0$, доказав таким образом теорему.

\subsection{Базис Гребнера}
В предыдущем параграфе была приведена и доказана фундаментальная теорема, связанная с базисом идеалов в кольцах многочленов.
Далее  будет введено понятие базиса Гребнера и описаны его особенности.

Пусть задано мономиальное упорядочение. Конечное подмножество $G = g_{1}, ... , g_{s}$ элементов идеала $I$ называется его базисом Гребнера, если 
$$ (LT(g_{1}), ..., LT(g_{s})) = ( LT(I) ) $$

Стоит отметить, что любой $I \triangleleft K \left[ x_{1}, ... , x_{n} \right]$ имеет базис Гребнера. Это легко увидеть из доказательства теоремы Гильберта о базисе.

Базис Гребнера является очень важным элементом в алгебраическом криптоанализе. Он помогает упрощать и решать алгебраические системы уравнений, выявлять различные особенности, связанные с той или иной системой алгебраических уравнений (например, существует эффективный критерий несовместимости системы). Поэтому, нахождение базиса Гребнера является одной из важных математических задач на данный момент. Далее будут рассмотрены три алгоритма нахождения базиса Гребнера (алгоритм Бухбергера, алгоритм $F4$ и алгоритм $F5$).

%%% Local Variables:
%%% mode: latex
%%% TeX-master: "rpz"
%%% End:
