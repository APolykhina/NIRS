% arara: pdflatex
% arara: bibtex
% arara: makeglossaries
% arara: pdflatex
% arara: remover: { patterns: [ '*.aux', '*.xdy', '*.toc', '*.out', '*.gl*', '*.ac*', '*.alg', '*.blg', '*.bbl' ] }
%% Преамбула TeX-файла

\documentclass[utf8, 14pt, bold]{G7-32}
\include{preamble.inc}

\ifPDFTeX
	\include{listings.inc}
\else
	\usepackage{local-minted}
\fi

\include{macros.inc}
\renewcommand{\thesection}{\arabic{section}}

\begin{document}
\frontmatter

\RPZ{к научно-исследовательской работе}
\RPZTheme{\null\hfill \bigskip Исследование методов решения проблемы \hfill\null \\ \\ \null\hfill \bigskip миллионеров Яо и вариантов применения \hfill\null \\ \\ \null\hfill \bigskip этих решений в тематике финансовых \hfill\null \\ \\ \null\hfill \bigskip и экономических задач \hfill\null} % ugly
\RPZGroup{ИУ8-94}
\RPZStudent{А.Д. Егорова}
\RPZTeacher{Д.А. Жуков}


\StrLen{\RPZTitleStudent}[\StudentLen]
\StrLen{\RPZTitleTeacher}[\TeacherLen]
\ifthenelse{\StudentLen > \TeacherLen}{\StrLen{\RPZTitleStudent}[\SignLen]}{\StrLen{\RPZTitleTeacher}[\SignLen]}

\pagestyle{empty}
\noindent
\begin{tabularx}{\textwidth}{L{2cm} C{\textwidth - 2.7cm}}
    \includegraphics[scale=1]{inc/img/bmstu.jpg} &
    {\centering\rmfamily\bfseries\fontsize{11pt}{11pt}\selectfont
    Министерство науки и высшего образования Российской Федерации \\
    Федеральное государственное бюджетное образовательное учреждение \\
    высшего образования \\
    <<Московский государственный технический университет \\
    имени Н.Э. Баумана \\
    (национальный исследовательский университет)>>\\
    (МГТУ им. Н.Э. Баумана)}\\
    \bottomrule[2pt]
    \bottomrule
\end{tabularx}

\noindent
\begin{tabularx}{\textwidth}{lX}
{\fontsize{12pt}{12pt}\selectfont ФАКУЛЬТЕТ}
    &
    {\fontsize{12pt}{12pt}\selectfont <<Информатика и системы управления>> (ИУ)}\\
    \hhline{~~}
    {\fontsize{12pt}{12pt}\selectfont КАФЕДРА} &
    {\fontsize{12pt}{12pt}\selectfont <<Информационная безопасность>> (ИУ8)}\\
    \hhline{~~}
\end{tabularx}
\vfill


\noindent
{\centering\rmfamily\bfseries\fontsize{20pt}{20pt}\selectfont
РАСЧЕТНО-ПОЯСНИТЕЛЬНАЯ ЗАПИСКА \\ \bigskip
\MakeTextUppercase{\RPZTitle} \\ \bigskip
НА ТЕМУ: \\ \bigskip
}


\vspace{1cm}

\noindent
\adjustbox{minipage=\textwidth}{\centering\bfseries\fontsize{20pt}{30pt}\selectfont \emph{\RPZTitleTheme}}

\vfill

\noindent
{\rmfamily\fontsize{12pt}{12pt}\selectfont Студент} \hfill $\stackrel[\text{\rmfamily\fontsize{9pt}{9pt}\selectfont (Группа)}]{}{\uline{\text{\rmfamily\fontsize{12pt}{12pt}\selectfont \RPZTitleGroup}}}$ \hfill\null \Signature{\RPZTitleStudent}


\noindent
{\rmfamily\fontsize{12pt}{12pt}\selectfont Научный руководитель} \hfill \Signature{\RPZTitleTeacher}

\vfill

{\centering\rmfamily\itshape\fontsize{14pt}{14pt}\selectfont \the\year~г. \par}
\newpage
\pagestyle{plain}

\Define{MAX}{максимум}
\Define{API}{test}

\newacronym{lvm}{LVM}{Logical Volume Manager}

\newacronym{test}{ТЕСТ}{Logical Volume Manager}

\include{00-abstract}

\tableofcontents
\printglossary[title=ТЕРМИНЫ И ОПРЕДЕЛЕНИЯ, toctitle=ТЕРМИНЫ И ОПРЕДЕЛЕНИЯ,nonumberlist]
\printglossary[type=acronym, title=ПЕРЕЧЕНЬ СОКРАЩЕНИЙ И ОБОЗНАЧЕНИЙ, toctitle=ПЕРЕЧЕНЬ СОКРАЩЕНИЙ И ОБОЗНАЧЕНИЙ,nonumberlist]

\Introduction

Целью работы является создание всякой всячины. Для достижения поставленной цели необходимо решить следующие задачи:

\begin{itemize}
\item проанализировать существующую всячину;
\item спроектировать свою, новую всячину;
\item изготовить всякую всячину;
\item проверить её работоспособность.
\end{itemize}

Проверяем как у нас работают сокращения, обозначения и определения "---
\gls{MAX}
API 
\gls{API}
\gls{test}
с обратным прокси.




\chapter*{ОСНОВНАЯ ЧАСТЬ}
\addcontentsline{toc}{chapter}{ОСНОВНАЯ ЧАСТЬ}
tests
\mainmatter

\input{20-analysis}
\input{30-design}
%\input{40-impl}
\input{50-research}
\input{60-economics}
\input{70-bzd}

\backmatter %% Здесь заканчивается нумерованная часть документа и начинаются ссылки и
            
\input{80-conclusion}%% заключение


\input{81-biblio}


\appendix   % Тут идут приложения

\input{90-appendix1}
\input{91-appendix2}

\end{document}