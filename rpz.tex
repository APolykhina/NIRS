% arara: pdflatex
% arara: bibtex
% arara: makeglossaries
% arara: pdflatex
% arara: remover: { patterns: [ '*.aux', '*.xdy', '*.toc', '*.out', '*.gl*', '*.ac*', '*.alg', '*.blg', '*.bbl' ] }
%% Преамбула TeX-файла

\documentclass[utf8, 14pt, bold]{G7-32}
\include{preamble.inc}

\ifPDFTeX
	\include{listings.inc}
\else
	\usepackage{local-minted}
\fi

\include{macros.inc}
\renewcommand{\thesection}{\arabic{section}}

\begin{document}
\frontmatter

\RPZ{к научно-исследовательской работе}
\RPZTheme{\null\hfill \bigskip Современные методы  \hfill\null \\ \\ \null\hfill \bigskip алгебраического криптоанализа \hfill\null \\ \\}%ugly
\RPZGroup{ИУ8-94}
\RPZStudent{А.Д. Полухина}
\RPZTeacher{П.Г. Ключарев}


\StrLen{\RPZTitleStudent}[\StudentLen]
\StrLen{\RPZTitleTeacher}[\TeacherLen]
\ifthenelse{\StudentLen > \TeacherLen}{\StrLen{\RPZTitleStudent}[\SignLen]}{\StrLen{\RPZTitleTeacher}[\SignLen]}

\pagestyle{empty}
\noindent
\begin{tabularx}{\textwidth}{L{2cm} C{\textwidth - 2.7cm}}
    \includegraphics[scale=1]{inc/img/bmstu.jpg} &
    {\centering\rmfamily\bfseries\fontsize{11pt}{11pt}\selectfont
    Министерство науки и высшего образования Российской Федерации \\
    Федеральное государственное бюджетное образовательное учреждение \\
    высшего образования \\
    <<Московский государственный технический университет \\
    имени Н.Э. Баумана \\
    (национальный исследовательский университет)>>\\
    (МГТУ им. Н.Э. Баумана)}\\
    \bottomrule[2pt]
    \bottomrule
\end{tabularx}

\noindent
\begin{tabularx}{\textwidth}{lX}
{\fontsize{12pt}{12pt}\selectfont ФАКУЛЬТЕТ}
    &
    {\fontsize{12pt}{12pt}\selectfont <<Информатика и системы управления>> (ИУ)}\\
    \hhline{~~}
    {\fontsize{12pt}{12pt}\selectfont КАФЕДРА} &
    {\fontsize{12pt}{12pt}\selectfont <<Информационная безопасность>> (ИУ8)}\\
    \hhline{~~}
\end{tabularx}
\vfill


\noindent
{\centering\rmfamily\bfseries\fontsize{20pt}{20pt}\selectfont
РАСЧЕТНО-ПОЯСНИТЕЛЬНАЯ ЗАПИСКА \\ \bigskip
\MakeTextUppercase{\RPZTitle} \\ \bigskip
НА ТЕМУ: \\ \bigskip
}


\vspace{1cm}

\noindent
\adjustbox{minipage=\textwidth}{\centering\bfseries\fontsize{20pt}{30pt}\selectfont \emph{\RPZTitleTheme}}

\vfill

\noindent
{\rmfamily\fontsize{12pt}{12pt}\selectfont Студент} \hfill $\stackrel[\text{\rmfamily\fontsize{9pt}{9pt}\selectfont (Группа)}]{}{\uline{\text{\rmfamily\fontsize{12pt}{12pt}\selectfont \RPZTitleGroup}}}$ \hfill\null \Signature{\RPZTitleStudent}


\noindent
{\rmfamily\fontsize{12pt}{12pt}\selectfont Научный руководитель} \hfill \Signature{\RPZTitleTeacher}

\vfill

{\centering\rmfamily\itshape\fontsize{14pt}{14pt}\selectfont \the\year~г. \par}
\newpage
\pagestyle{plain}

\Define{MAX}{максимум}
\Define{API}{test}

\newacronym{lvm}{LVM}{Logical Volume Manager}

\newacronym{test}{ТЕСТ}{Logical Volume Manager}

\include{00-abstract}

\tableofcontents
\printglossary[title=ТЕРМИНЫ И ОПРЕДЕЛЕНИЯ, toctitle=ТЕРМИНЫ И ОПРЕДЕЛЕНИЯ,nonumberlist]
\printglossary[type=acronym, title=ПЕРЕЧЕНЬ СОКРАЩЕНИЙ И ОБОЗНАЧЕНИЙ, toctitle=ПЕРЕЧЕНЬ СОКРАЩЕНИЙ И ОБОЗНАЧЕНИЙ,nonumberlist]

\Introduction

Целью работы является создание всякой всячины. Для достижения поставленной цели необходимо решить следующие задачи:

\begin{itemize}
\item проанализировать существующую всячину;
\item спроектировать свою, новую всячину;
\item изготовить всякую всячину;
\item проверить её работоспособность.
\end{itemize}

Проверяем как у нас работают сокращения, обозначения и определения "---
\gls{MAX}
API 
\gls{API}
\gls{test}
с обратным прокси.





\chapter*{ОСНОВНАЯ ЧАСТЬ}
\addcontentsline{toc}{chapter}{ОСНОВНАЯ ЧАСТЬ}
\mainmatter

%\input{40-impl}
\section{Понятие идеала в кольце многочленов}
\label{cha:basedifinition}

В первой части научно-исследовательской работы даются базовые определения и понятия, которые будут использоваться на протяжении всей расчетно-пояснительной записки.
\\Пусть $R$ - коммутативное кольцо с единицей $1$.


\subsection{Понятие идеала}

Непустое подмножество $I$ кольца $R$ называется идеалом в $R$ (записывается $I \triangleleft R$), если:

1) для любых элементов $a,b \in I$ элемент $a + b \in I$;

2)для любых $a \in I, c \in R$ элемент $ac \in I$.

Идеал $I$ кольца $R$ называется главным, если существует такой элемент $a \in I$, что $ I = (a)$. Элемент $a$ называется порождающим( или образующим)  для идеала $I$.

Кольцо  $R$ называется кольцом главных идеалов, если каждый идеал кольца $R$ является главным.

\subsection{Базис идеала}

Обобщим по понятие главного идеала $I$ кольца $R$.  Пусть 
$a_{1}, ... , a_{k}$ - произвольные элементы кольца $R$.

Если множество 
$$ (a_{1}, ... , a_{k}) =\left\{ a_{1}r_{1} + ... + a_{k}r_{k};  r_{1}, ... , r_{k} \in R \right\} $$ есть идеал $I$ кольца $R$, тогда говорят, что элементы $a_{1}, ... , a_{k}$ составляют базис идеала $I$. Обозначается  $I = (a_{1}, ... , a_{k})$.

Можно заметить, что при определении базиса идеала нет требования на минимальное количество базисных элементов. Это связано с тем, что при добавлении к базису произвольного элемента идеала, мы получаем тот же самый базис.

Для любого элемента $a \in (a_{1}, ..., a_{k})$:
$$ (a_{1}, .. , a_{k}, a) = (a_{1}, ... ,a_{k}) $$ 

Легко убедиться в достоверности этого утверждения. По определению $ (a_{1}, ... , a_{k}, a) = \left\{ a_{1}r_{1} + ... + a_{k}r_{k} + ar;   r_{1}, ... , r_{k}, r \in R \right\} $. По определению идеала $I$ кольца $R$, если $a \in I, r \in R$, то элемент $ar \in I$. Тогда 
$ar = a_{1}r_{1}' + ... + a_{k}r_{k}';  r_{1}', ... , r_{k}', \in R $. Подставим это выражение в определение базиса $ (a_{1}, ... , a_{k}, a) = \left\{ a_{1}r_{1} + ... + a_{k}r_{k} + a_{1}r_{1}' + ... + a_{k}r_{k}';  r_{1}, r_{1}', ... , r_{k},  r_{k}' \in R \right\} = 
 \left\{ a_{1}(r_{1} + r_{1}') + ... + a_{k}(r_{k} + r_{k}');  r_{1}, r_{1}', ... , r_{k}, r_{k}' \in R \right\} $. Таким образом, при добавлении нового базисного элемента, который принадлежит идеалу, базис не изменится.
 
 \subsection{Идеал в кольцах многочленов}
 
 Пусть $K \left[ x_{1}, ... , x_{n} \right]$ - кольцо многочленов от переменных $x_{1}, ... ,x_{n}$ над полем $\mathcal{K}$.
 Рассмотрим идеалы в кольцах многочленов.
 
 Легко убедиться в том, что кольцо многочленов от нескольких переменных не является кольцом главных идеалов.
В кольце многочленов $K \left[ x_{1}, ... , x_{n} \right]$ выделим множество многочленов, у которых свободных член равен $0$. Данное множество образует идеал в этом кольце, обозначим его $I_{0}$. Пусть $ I_{0}  = (f), f \in K \left[ x_{1}, ... , x_{n} \right] $. Поскольку $x_{1} \in I_{0}$, значит $f$ - либо ненулевая константа (тогда $I_{0} =K \left[ x_{1}, ... , x_{n} \right] $, что является противоречием условия), либо $f = ax, a \in \mathcal{K}, a \neq 0$. Но $x_{2} \in I_{0}$, значит $f$ делит 
 $x_{2}$. Возникает противоречие.
 
Раз кольцо многочленов от нескольких переменных не является кольцом главных идеалов, значит имеет место понятие базиса идеала. В следующем параграфе будут подробно рассмотрены базисы идеала в кольце многочленов от нескольких переменных и выведены некоторые утверждения.







 
 
 


%%% Local Variables:
%%% mode: latex
%%% TeX-master: "rpz"
%%% End:

\section{Мономиальные базисы идеалов в кольцах многочленов} 

В данном разделе будут подробно рассмотрены мономиальные базисы идеалов в кольцах многочленов от нескольких переменных и введено понятие базиса Гребнера.

\subsection{Мономиальный порядок и его свойства}
Пусть $ M_{n} $ - множество мономов. 
Мономиальным упорядочением $ \prec $ на $ M_{n} $ называется линейный порядок, удовлетворяющий свойствам:

1)  $M \prec N \Rightarrow MP \prec NP,  \forall M, N, P \in M_{n}$ 

2)$ 1 \preceq M,  \forall M \in M_{n} $

Пусть $ M = x_{1}^{a_{1}}...x_{n}^{a_{n}} $, $ N = x_{1}^{b_{1}}...x_{n}^{b_{n}} $ - произвольные мономы из $ M_{n} $.
Приведем несколько примеров бинарных отношений на $M_{n}$, которые являются мономиальными упорядочениями:

1.  Лексикографическое упорядочение (lex):
$$ M \prec_{lex} N \Leftrightarrow (a_{1}, ... , a_{n}) \prec_{lex} (b_{1}, ..., b_{n}) $$

2. Сначала по степени, затем лексикографическое упорядочение (deglex):
$$ M \prec_{deglex} N \Leftrightarrow (deg(M), a_{1}, ... , a_{n}) \prec_{lex} (deg(N), b_{1}, ..., b_{n}) $$

3. Сначала по степени, затем обратное лексикографическое упорядочение (degrevlex):
$$ M \prec_{degrevlex} N \Leftrightarrow (deg(M), b_{n}, ... , b_{1}) \prec_{lex} (deg(N), a_{n}, ..., a_{1}) $$

Далее введем несколько обозначений, которые понадобятся нам в дальнейшем.

Пусть $ f= \sum_{a}p_{a}x^{a}$ - ненулевой полином в $K \left[ x_{1}, ... , x_{n} \right]$, и пусть $ \prec $ - мономиальное упорядочение.

1. Мультистепень полинома $f$ определяется так:
$$ multideg(f) = max(a \in Z_{\geq0}^{n}: p_{a} \neq 0) $$

2. Старший коэффициент полинома $f$:
$$ LC(f) = p_{multideg}(f) $$

3. Старший моном полинома $f$:
$$ LM(f) = x^{multideg}(f) $$ ( с коэффициентом 1)

4. Старший член полинома $f$:
$$ LT(f) = LC(f)LM(f) $$ 

\subsection{Мономиальные идеалы}

Введем понятие мономиального идеала в кольцое многочленов от нескольких переменных.

Идеал $I \triangleleft K \left[ x_{1}, ... , x_{n} \right]$ называется мономиальным, если существует подмножество $A \in Z_{\geq0}^{n}$ (которое может быть бесконечым), такое, что $I$ состоит из всех конечных сумм вида $\sum_{a \in A}h_{a}x^{a}$, где $h_{a} \in  K \left[ x_{1}, ... , x_{n} \right]$. Такой идеал будем обозначать $(x^{a}: a \in A)$.

\textbf{Лемма Диксона.} {
\it 
Любой мономиальный идеал $ I = ( x^{a} : a \in A)$ может быть представлен в виде $ I = (x^{a(1)}, ... , x^{a(s)}$, где $a(1), ..., a(s) \in A$. В частности, $I$ имеет конечный базис.
}

\subsection{Теорема Гильберта о базисе}
В предыдущем параграфе было  показано, что любой мономиальный идеал кольца многочленов от нескольких переменных иметь конечный базис. В этом разделе будет доказано, что любой идеал в кольце многочленов(не только мономиальный) имеет конечный базис.

\textbf{Теорема Гильберта о базисе.}{
\it
Каждый идеал $I \triangleleft K \left[ x_{1}, ... , x_{n} \right]$ является конечно порожденным, то есть $I = (g_{1}, ..., g_{s})$, где $g_{1}, ..., g_{s} \in I$.
}

\textbf{Доказательство.} 
Пусть $I \triangleleft K \left[ x_{1}, ... , x_{n} \right]$. Обозначим через $a_{t}$ множество элементов $a \in \mathcal{K}$, которое содержит в себе $LC(f)$, где $f \in I$ и $ multideg(f) = t $.
Легко видно, что $a_{i}$ является идеалом в $ \mathcal{K} $. (Если $a,b \in a_{i}$, то $ a \pm b \in a_{i}$, это легко увидеть, достаточно взять сумму или разность соответсвующих многочленов. Если $ c \in \mathcal{K}$, то $ca \in a_{i}$, это можно увидеть при умножении соответсвующего многочлена на $c$.)
Кроме того, имеем $$ a_{0} \subset a_{1} \subset a_{2} ... $$
другими словами последовательность идеалов $\left[ a_{i} \right]$ возрастающая. 
\\Пусть последовательность идеалов стабилизируется на $a_{r}$:
$$ a_{0} \subset a_{1} \subset a_{2} ... \subset a_{r} = a_{r+1} = ...$$
Пусть 
$$  a_{0}  = (a_{01}, a_{02}, ..., a_{0n_{0}})$$
$$  a_{r}  = (a_{r1}, a_{r2}, ..., a_{rn_{0}})$$
Для каждого $ i = 0, ...., r$ и $j = 0, ..., n_{i}$ пусть $f_{ij}$ - многочлен из $I$ степени $i$ со старшим коэффициентом $a_{ij}$. Покажем, что многочлены $f_{ij}$ образуют базис для $I$.
\\ Пусть $f$ - многочлен степени $d$ из $I$. Индукцией по $d$ докажем, что $f$ лежит в идеале, порожденном $f_{ij}$.
\\ Если $d > r$, то заметим, что старшие коэффициенты многочленов 
$$ X^{d-r}f_{r1}, ... , X^{d-r}f_{rn_{r}} $$
порождают $a_{d}$. Значит, существуют $ c_{1}, ... , c_{n_{r}} \in  \mathcal{K}$, такие, что многочлен
$$ f - c_{1}X^{d-r}f_{r1} - ... - c_{n_{r}}X^{d-r}f_{rn_{r}} $$
имеет степень $ < d$. Этот многочлен лежит в $I$.
Если $ d \leq r$, мы можем получить многочлен степени $ < d$, лежащий в $I$, вычитая некоторую линейную комбинацию
$$ f - c_{1}f_{d1} - ... - c_{n_{r}}f_{dn_{r}} $$
По индукции мы можем найти такой многочлен $g \in I$, который порожден $f_{ij}$ и $f - g = 0$, доказав таким образом теорему.


%%% Local Variables:
%%% mode: latex
%%% TeX-master: "rpz"
%%% End:

\backmatter %% Здесь заканчивается нумерованная часть документа и начинаются ссылки и
            
\input{80-conclusion}%% заключение


\input{81-biblio}


\appendix   % Тут идут приложения

\input{90-appendix1}
\input{91-appendix2}

\end{document}